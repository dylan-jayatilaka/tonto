\documentclass{article}

\usepackage{graphicx}
\usepackage[dvipsnames]{xcolor}
\usepackage{tikz}
\usepackage{mol2chemfig}
\usepackage{chemfig}
\usepackage{longtable}

% Colors
\usetikzlibrary{calc}
\colorlet{edge-color}{blue!50}
\colorlet{fill-color}{blue!20}
\colorlet{c-color}{black!100}
\colorlet{i-color}{black!100}
% \colorlet{c-color}{RawSienna}
% \colorlet{i-color}{RedOrange}
\newcommand\percentc{\%}

% Bond dial diagram macro
% #1 - atom label 1
% #2 - atom label 2
% #3 - total bond index
% #4 - covalent index
% #5 - ionic    index
% #6 - c/i list, normalised so \r=1
% #7 - overall scale divisor
\newcommand{\dialdiagram}[7]{

\begin{tikzpicture}[scale=#7]

% \draw[help lines, color=gray!30] (0,0) grid (#1,#1);

\pgfmathsetmacro\tot{#3}
\pgfmathsetmacro\cov{#4}
\pgfmathsetmacro\ion{#5}
\pgfmathsetmacro\r{1.0} % Normalised to 1
\pgfmathsetmacro\s{1.05*\r}
\pgfmathsetmacro\q{0.9*\r}
\pgfmathsetmacro\t{0.05*\r}
\pgfmathsetmacro\b{0.40*\r}

% Get tip position from c/i list #4
\pgfmathsetmacro{\c}{0}
\pgfmathsetmacro{\i}{0}
\foreach \x/\y in {#6}{ 
   \pgfmathparse{\c+\x}\xdef\c{\pgfmathresult}
   \pgfmathparse{\i+\y}\xdef\i{\pgfmathresult}
}

% Axes with atom labels #1, #2
\draw[color=gray!60,->] (0,0)--(0,\r) ; % y axis
\draw[color=gray!60,->] (0,0)--(\r,0) ; % x axis

% Draw individual (c,i) bonds stored in #6
\draw[color=edge-color,thin,fill=fill-color]
   (0,0)
   foreach \x/\y in {#6}
      { -- ++(\x,\y) } -- (\c,0) -- cycle ;

% Draw circles at (c,i) bonds stored in #6
\draw[fill=fill-color]
   (0,0)
   foreach \x/\y in {#6}
      { ++(\x,\y) circle[radius=0.01] } ;

% Bond indices
\node[color=c-color,below] at (\c,-\t) {{$\cov$}} ;
\node[color=i-color,left]  at (0 , \i) {{$\ion$}} ;
\node[above] at (0, \s) {{$\tau_\textrm{\tiny #1-#2}=\tot$}} ;

% c & i index line
% \draw[color=gray!60,dotted] (\c,0)--(\c,\i) ;
\draw[color=gray!60,dotted] (0,\i)--(\c,\i) ;

% Atom labels
% \node[left]       at (0,\r) {{\scriptsize #1}} ;
% \node[below left] at (0, 0) {{\scriptsize #2}} ;

% Quadrant arc of radius \r
\draw[gray!60] (\r,0) arc [start angle=0, end angle=90, radius=\r] ;
\foreach \angle in {9, 18, 27, 36, 45, 54, 63, 72, 81}
   { \draw[gray!60] (\angle:\q) -- (\angle:\r); } ;

\end{tikzpicture}
}

\pagestyle{empty}

